\documentclass{book}

% Page setup
\setlength{\textheight}{8.5in}
\setlength{\textwidth}{6in}
\setlength{\topmargin}{-0.375in}
\setlength{\oddsidemargin}{.25in}
\setlength{\evensidemargin}{.25in}
\setlength{\headheight}{0.200in}
\setlength{\headsep}{0.4in}
\setlength{\footskip}{0.500in}
\setlength{\parskip}{1ex}
\setlength{\parindent}{0pt}
%\flushbottom

\usepackage[utf8]{inputenc}

\usepackage{graphicx, caption, subcaption}  % \includegraphics, \subfigures
\usepackage{tikz, pgfplots} % tikz and pgfplots
\pgfplotsset{compat=1.11}
\usepackage{tkz-euclide,tkz-fct} % helpful macros for plotting tikz
\usepackage{amsmath, amssymb, bm}
\usepackage{amsthm}
\usepackage{mathtools}
\usepackage{hyperref}
\hypersetup{colorlinks = true,allcolors=blue}
\usepackage{float}
\usepackage{xcolor}
\usepackage{enumitem}

\usepackage{tikz, pgfplots,tkz-euclide, tkz-fct} % tikz and pgfplots


\pgfplotsset{compat=1.11} % we dont have to use axis cs to merge tikz marking with pgfplots

\tikzset{insert node/.style args={#1 at #2}{
    postaction=decorate,
    decoration={
      markings,
      mark= at position #2
        with
        {
         #1
        }
    }
  }
}


\newcommand{\fig}[1]{Fig.~\ref{#1}}
\newcommand{\tbl}[1]{Table~\ref{#1}}
\newcommand{\algo}[1]{Algorithm~\ref{#1}}
\newcommand{\sect}[1]{Sec.~\ref{#1}}

\newtheorem{lemma}{Lemma}
\newtheorem{theorem}{Theorem}
\newtheorem{corollary}{Corollary}
\newtheorem{proposition}{Proposition}
\newtheorem{question}{Question}
\newtheorem{remark}{Remark}
\newtheorem{conjecture}{Conjecture}
\newtheorem{example}{Example}
\newtheorem{assumption}{Assumption}
\newtheorem{definition}{Definition}


\newcount\Comments  % 0 suppresses notes to selves in text
\Comments=1 % TODO: change to 0 for final version
\definecolor{darkgreen}{rgb}{0,0.5,0}
\definecolor{darkred}{rgb}{0.7,0,0}
\definecolor{teal}{rgb}{0.3,0.8,0.8}
\definecolor{orange}{rgb}{1.0,0.5,0.0}
\definecolor{purple}{rgb}{0.8,0.0,0.8}
\newcommand{\kibitz}[2]{\ifnum\Comments=1{\textcolor{#1}{\textsf{\footnotesize #2}}}\fi}
\newcommand{\anirudh}[1]{\kibitz{purple}{[AV: #1]}}
\newcommand{\alex}[1]{\kibitz{purple}{[AS: #1]}}
\newcommand{\akash}[1]{\kibitz{blue}{[Ak: #1]}}
\newcommand{\brady}[1]{\kibitz{green}{[BM: #1]}}



% mathbb aliases
\newcommand{\Real}{\mathbb{R}}
\newcommand{\Expec}{\mathbb{E}}
\newcommand{\Prob}{\mathbb{P}}
\newcommand{\Sphere}{\mathbb{S}}

% mathsf aliases
\newcommand{\projection}{\mathsf{projection}}
\newcommand{\normal}{\mathcal{N}}
\newcommand{\dataset}{\mathcal{D}}
\newcommand{\loss}{\mathcal{L}}
\newcommand{\continuous}{\mathcal{C}}

% Stupid aliases
\newcommand{\pandq}{~p~\text{and}~q}
\newcommand{\coeff}{~\text{coefficients of}}
\newcommand{\vectheta}{\vec{\theta}}
\newcommand{\vecp}{\vec{p}}
\newcommand{\vecl}{\vec{l}}
\newcommand{\vecw}{\vec{w}}
\newcommand{\vecu}{\vec{u}}
\newcommand{\vecx}{\vec{x}}
\newcommand{\vecN}{\vec{N}}

\title{16-811 Math Fundamentals for Robotics}
\author{Michael Erdmann}
\date{
Digitization Contributors: \\
Thomas Weng \\
Dominic Guri \\
Brady Moon \\
Roshni Kaushik \\
Anirudh Vemula \\
Ratnesh Madaan \\
Akash Sharma \\
Alex Spitzer \\
Ankit Bhatia 
}

\begin{document}

\frontmatter
    \maketitle
    \tableofcontents
    
\mainmatter
        \chapter{Introduction}
    
    Typed version of notes found here \url{http://www.cs.cmu.edu/~me/811/notes/handouts.html}
    \chapter{Solving Linear Equations}

We have discussed general methods for solving arbitrary equations. For the most part, we restricted our attention to the one-dimensional case. We also looked at a special class of equations, namely those given by polynomials. Another special class is given by linear equations. Of course, the one-dimensional case is absolutely trivial, so one naturally considers systems of equations. 

\noindent I assume that you are familiar with the basic results of linear algebra, such as the diagonalization theorems and Gaussian Elimination. This section provides a very quick review, then discusses Singular Value Decomposition (SVD).

\medskip
\noindent \textbf{Applications}: Applications for linear equations are numerous. These include, of course, Linear Programming problems. Other are the local solution of non-linear differential equations via linearization, and the determination of minima in least-squares problems. 

\newpage

\section{A quick review of important definitions and facts}

\begin{enumerate}
\item Given a linear function $f: \mathbb{R}^n \rightarrow \mathbb{R}^m$ and a pair of bases,
    \begin{align*}
        \mathcal{B}_1 &= {e_1,\dots, e_n}\text{ for }\mathbb{R}^n \\
        \mathcal{B}_2 &= {d_1,\dots, d_m}\text{ for }\mathbb{R}^m
    \end{align*}
    we can represent $f$ by an $m\times n$ matrix $A$.

    Notice that the matrix $A$ depends on the choice of bases $\mathcal{B}_1 \times \mathcal{B}_2$.
    
    Often we just let $\mathcal{B}_1 \times \mathcal{B}_2$ be the "obvious" bases for $\mathbb{R}^n \times \mathbb{R}^m$.
    
    \textbf{Moral:} A matrix is a particular coordinate representation of the linear function $f$.
    
\item Given an $m \times n$ matrix $A$, we make the following definitions:
    \begin{align*}
        \text{Column space: } & \text{linear combinations of the columns of }A\\
        \text{Row space: } & \text{linear combinations of the rows of }A
    \end{align*}
    If we think of $A$ as defining the linear mapping 
    \begin{align*}
        A: &\mathbb{R}^n \rightarrow \mathbb{R}^m \\
        &
        \begin{bmatrix} x_1 \\ \vdots \\ x_n \end{bmatrix}
        \rightarrow
        \begin{bmatrix} y_1 \\ \vdots \\ y_m \end{bmatrix}
        =
        A
        \begin{bmatrix} x_1 \\ \vdots \\ x_n \end{bmatrix}
    \end{align*}
    then the column space of $A$ is a vector subspace of $\mathbb{R}^m$, consisting of all points in $\mathbb{R}^m$ that are image vectors under $A$.
    
    Note that the row space of $A$ is just the column space of $A^T$, the transpose of $A$.
    
    And we define
        \begin{align*}
            \text{null space: } & \text{set of vectors }x\text{ in }\mathbb{R}^n\text{ such that }Ax=0.
        \end{align*}
    The following relationships are useful to remember:
    \begin{itemize}
        \item $\dim(\text{row space}) = \dim(\text{column space})$, either of which is called the \underline{rank} of $A$,
        \item The row space and null space are complementary (perpendicular) subspaces of $\mathbb{R}^n$. In other words, 
        \begin{align*}
            \mathbb{R}^n &= \text{row space}\oplus\text{null space} \\
            n &= \dim(\text{row space}) + \dim(\text{null space})
        \end{align*}
    \end{itemize}
    In picture form:
    \begin{figure}[H]
        \centering
        \includegraphics[width=0.9\textwidth]{figures/linear-1.png}
        % \caption{Caption}
        \label{fig:linear-1}
    \end{figure}
    
\item Suppose $A$ is an $m\times n$ matrix, and consider the system of equations 
    \begin{align*}
        Ax = b.
    \end{align*}
    \begin{itemize}
        \item If $b$ is not an element of the column space of $A$, then we say that the system is \underline{inconsistent}(or \underline{overdetermined}). 
        \item If $b$ is in the column space of $A$ and the null space of $A$ is non-trivial, then we say that the system is \underline{underdetermined}. In this case there is a whole family of solutions, given by the affine set
            $$x_0 + N$$
        where $x_0$ is any particular solution $Ax_0 = b$, and $N$ is the null space of $A$. 
        \item If $A$ is a $n\times n$ square matrix we say that $A$ is \underline{singular}
            \begin{description}
                \item \quad iff $\det(A) = 0$
                \item \quad iff $\text{rank}(A) < n$
                \item \quad iff the rows of $A$ are not linearly independent
                \item \quad iff the columns of $A$ are not linearly independent
                \item \quad iff the dimension of the null space of $A$ is non-zero
                \item \quad iff A is not invertible.
            \end{description}
    \end{itemize}
\end{enumerate}

\newpage
\section{A quick review of matrix decompositions}
(we will look at SVD in more detail)

\subsection{Factorizations based on elimination}

Given an $m\times n$ matrix $A$ (with $m \geq n = \text{rank}(A)$), we can write $A$ in the form 
\begin{align*}
    PA &= LDU
\end{align*}
where 
\begin{description}
    \item \quad $P$ is an $m\times m$ permutation matrix that specifies row interchanges, 
    \item \quad $L$ is an $m\times m$ square low-triangular matrix with 1's on the diagonal, 
    \item \quad $U$ is an $m\times n$ upper-triangular matrix with 1's on the diagonal, and 
    \item \quad $D$ is an $m\times m$ square diagonal matrix.
\end{description}

[More generally: we may also need to perform column interchanges on $A$, in which case we get two permutation matrices, so $P_1AP_2 = LDU$.]

Notes:
\begin{enumerate}
    \item The entries on the diagonal of $D$ are sometimes called ``pivots" (after the Gaussian Elimination algorithm). 
    \item The product of the pivots is equal to $\pm\det(A)$ (sign depends on $P$; negative if odd row interchanges), whenever $A$ is a square matrix.
    \item If $A$ is symmetric and $P=I$, then $U=L^T$.
    \item If $A$ is symmetric positive-definite, then $U=L^T$ and the diagonal entries of $D$ are strictly positive.
\end{enumerate}

\underline{Ex}
\begin{enumerate}[label=(a)]
\item 
    \begin{align*}
        \begin{bmatrix}
            1 & 0 \\
            1 & 1 \\
            0 & -1
        \end{bmatrix}
        &=
        \begin{bmatrix}
            1 & 0 & 0 \\
            1 & 1 & 0 \\
            0 & -1 & 1
        \end{bmatrix}
        \begin{bmatrix}
            1 & 0 & 0 \\
            0 & 1 & 0 \\
            0 & 0 & 1
        \end{bmatrix}
        \begin{bmatrix}
            1 & 0 \\
            0 & 1 \\
            0 & 0
        \end{bmatrix}
    \end{align*}
    
    Note: can use any value in the last entry of $D$. Use $0$ if we want $\text{rank}(D) = \text{rank}(A)$.
    
    [If $m > n$, as in this example, then the last $m-n$ rows of $U$ are zero.]

\item
        \begin{align*}
        \begin{bmatrix}
            1 & 1 & 0 \\
            2 & 1 & -1 \\
        \end{bmatrix}
        &=
        \begin{bmatrix}
            1 & 0 \\
            2 & 1
        \end{bmatrix}
        \begin{bmatrix}
            1 & 0 \\
            0 & -1 
        \end{bmatrix}
        \begin{bmatrix}
            1 & 1 & 0 \\
            0 & 1 & 1
        \end{bmatrix}
    \end{align*}
\end{enumerate}

\noindent \underline{Algorithm:} Gaussian Elimination directly yields this decomposition.

\medskip

\noindent \underline{Application:} As with most decompositions/factorizations, the hope is to simplify solving the system $Ax=b$. 

Suppose $A$ is square and non-singular. THen solving $Ax=b$ really means solving 
\begin{align*}
    LDUx &= Pb.
\end{align*}
In turn this entails solving two simpler problems: 
\begin{align*}
    &\text{(i) } Ly = Pb & \text{(solve for $y$)} \\
    &\text{(ii) } Ux = D^{-1}y & \text{(solve for $x$)}
\end{align*}
Each of these problems can be solved easily using forward or back substitution ($D^{-1}$ is easy to compute since $0$ is diagonal with non-zero entries).

\subsection{Factorizations based on eigenvalues}

These are the standard factorization s one learns in a linear algebra course. Two important ones are:

\begin{enumerate}
\item If $A$ is a square $n\times n$ matrix with $n$ linearly independent eigenvectors, then
    \begin{align*}
        A &= S\Lambda S^{-1},
    \end{align*}
    where $\Lambda$ is a diagonal matrix whose entries are the eigenvalues of $A$,
    
    \medskip
    
    and $S$ is a matrix whose columns are the eigenvectors of $A$.
    
    \medskip
    
    This factorization is not always possible. (One case in which it is possible ofccurs when $A$ has $n$ distinct eigenvalues.)

\item One can always decompose $A$ in Jordan form, i.e.,
    \begin{align*}
        A &= MJM^{-1}
    \end{align*}
    where $J = 
    \begin{bmatrix}
        J_1 & \dots & 0 \\
        \vdots & \ddots & \vdots \\
        0 & \dots & J_s
    \end{bmatrix}
    $ is a block matrix, such that each block \newline
    $J_i = 
    \begin{bmatrix}
        \lambda_i^1 & \dots & 0 \\
        \vdots & \ddots & \vdots \\
        0 & \dots & \lambda_i^s
    \end{bmatrix}
    $ with $\lambda_i$ an eigenvalue of $A$.

    \medskip
    
    Here $s$ is the number of independent eigenvectors of $A$. $M$ consists of eigenvectors and ``generalized" eigenvectors.

    \medskip
    
    [Side note: ]

\end{enumerate}

\subsection{Factorizations based on $A^TA$}

\subsubsection{QR}

\subsubsection{SVD}

\section{SVD in more detail}
    \chapter{Polynomial Approximations - Interpolation}
% Brady is currently working on this section

\brady{test}
        \chapter{Solution of Nonlinear Equations}
    \chapter{Roots of Polynomials}

So far we have looked at methods for finding roots of arbitrary functions. Now let us specialize these into polynomials. \newline

\noindent \emph{Applications}: Why polynomials?
\begin{itemize}
    \item We already saw the use of polynomials in approximation/interpolation. Finding their roots is a natural next step (we might have an approximate representation of a function whose roots we need, say, to ensure safety of a trajectory).
    \item Systems of multivariate polynomial equations have seen much attention recently, particularly in motion planning, and to some extent in grasping and in machine vision. ``Exact'' algebraic solution (i.e., root bracketing) are used to find roots. (c.f. Canny, Theory of the reals, etc.)
\end{itemize}

%\hline 

\vspace{1em}

\noindent We will only look at polynomials in one variable. \\
(Side note: Often, the solution of multi-variate polynomials is transformed into a problem requiring the solution of a single-variate polynomial.) \\

\noindent Reminder: 
\begin{itemize}
    \item Closed-form formulas exist for finding the roots of polynomials of degree 1, 2, 3, and 4.
    \item It is impossible to have a general formula for $\geq$ 5.
    \item Even degree 3 and 4 are somewhat unstable numerically.
\end{itemize}

\noindent Some facts: \\
Suppose  we write a polynomial of degree $n$ as
$$p(x) = a_n x^n + a_{n-1} x^{n-1} + \dots + a_1 x + a_0\text{, with }a_n \neq 0$$

\begin{enumerate}
    \item $p$ has $n$ real or complex roots, counting multiplicities. \\
    \item If the coefficients $\{ a_i \}$ are all real, then the complex roots occur in conjugate pairs.
    \item \emph{Descartes' rule of sign} 
\end{enumerate}


    \chapter{Resultants and Elimination Theory}

Worked on by Anirudh.

    \chapter{Approximation}
{\parindent0pt
\underline{Purpose}: Approximate a general function by a class of simpler functions

There are two motivations for this:
\begin{enumerate}
    \item Decompose a complicated function into its constituent simpler functions, in order to manipulate the function more easily (e.g. differentiation, integration, storage, etc.)
    \item Recover a function from partial or noisy information (for example, we may have stored some values of the function in a table, or we may be observing or sensing the function at discrete times
\end{enumerate}

\underline{Applications}:
\begin{itemize}
    \item Signal compression and reconstruction (Fourier techniques)
    \item Data fitting (best line, best quadratic, ...)
    \item CAD representations of shapes
\end{itemize}
\emph{Note: Approximation goes beyond interpolation}

\section{Uniform approximation by polynomials}
Let us start by looking at polynomials again. This time, rather than interpolate at given fixed points, we will seek the best uniform approximation. \\

Given a function $f:[a, b] \rightarrow \mathbb{R}$ and a polynomial $p$, we measure the error between $f$ and $p$ in terms of an $L_\infty$ norm, that is,
\begin{equation*}
    ||f - p||_\infty = \max_{a \leq x \leq b}{|f(x) - p(x)|}
\end{equation*}

A good uniform approximation is one for which this error is small. Recall from Weierstrass' Theorem that (for continuous f), we can make this error arbitrarily small by choosing a polynomial $p$ of high-enough degree. \\

Let's restrict the degree of the polynomial \\

\underline{Def:} Let \underline{$\Pi_n$} consist of all polynomials of degree at most $n$ \\

\underline{Def:} The \underline{uniform distance} of $f$ from $\Pi_n$ is the smallest error achievable using polynomials in $\Pi_n$.
\begin{equation*}
    d (f, \Pi_n) = \min_{p \in \Pi_n}{||f - p||_\infty}
\end{equation*}

The goal in best uniform approximation is to find a polynomial $p$ that actually achieves this minimum possible error. \\

The following theorem helps us: \\

\fbox{\parbox{\textwidth}{%
\underline{Theorem}: A function $f$ which is continuous on $[a, b]$ from $\Pi_n$ \\

The polynomial $p \in \Pi_n$ is the best uniform approximation to $f$ on $[a,b]$
\begin{align*}{\text{if and only if}}\end{align*}

There are $n + 2$ points $a \leq x_0 \leq ... \leq x_{n+1} \leq b$ such that \\

\begin{equation*}
    (-1)^2 [f(x_i) - p(x_i)] = \epsilon || f - p ||_\infty ~~~ i = 0, ..., n+1
\end{equation*}
where $\epsilon = signum[f(x_0) - p(x_0)]$\\

(Here $x_0 = a$ and $x_{n+1} = b$ in case $f^{(n+1)}(x)$ does not change sign on $[a,b]$)}%
}\\

In other words, with alternating sign at $n+2$ points, the difference between $f$ and $p$ is precisely equal to the $L_\infty$ distance between $f$ and $p$. \\

The idea is the apply this theorem by constructing a polynomial $p$ that satisfies the alternating-error-condition. By the theorem, that polynomial is the best uniform approximation. \\

\underline{Ex.} Consider $f(x) = e^{x}~\text{on}~[-1, 1]$ \\

Let us construct the best uniform approximation to $e^x$ by a straight line over the interval $[-1, 1]$. \\

With $n=1$, the alternating-error-condition of the previous theorem tells us that qualitatively the best uniform approximation will look something like this.

\begin{figure}[H]
    \centering
    \includegraphics[width=0.9\textwidth]{figures/approximation_1}
    % \caption{Caption}
    \label{fig:approximation-1}
\end{figure}

The point is that there are three points $x_0 = -1$, $x_1 = ?$, $x_2 = 1$ at which the error $f(x) - p(x)$ is greatest, with equal magnitude and alternating sign. \\

The trick is to figure out what the point $x_1$ is. To do this, let's try to figure out what the theorem tells is in detail. \\

Let's write the polynomial as $p(x) = a + bx$ \\

(since $f''(x)$ does not change sign on $[-1, 1]$, we know that $x_0 = -1$ and $x_2 = 1$) \\

Computing errors we find that:
\begin{align*}
    e(x_0) &= f(x_0) - p(x_0) = f(-1) - p(-1) = \frac{1}{e} - a + b\\
    e(x_1) &= f(x_1) - p(x_1) = f(x_1) - p(x_1) = e^{x_1} - a - bx_1\\
    e(x_2) &= f(x_2) - p(x_2) = f(1) - p(1) = e - a - b
\end{align*}

The theorem says $e(x_0) = - e(x_1) = e(x_2) = ||f - p||_\infty$
\begin{align*}
    e(x_0) &= e(x_2) \\
    \frac{1}{e} - a + b &= e - a - b \\
    2b &= e - \frac{1}{e} \\
    b &= \frac{e - \frac{1}{e}}{2} \approx 1.1752
\end{align*}

This says that the slope of our best line must be the same as the average change of $f(x)$ over $[-1,1]$. \\

\underline{How do we choose a?}

Well, by definition $||f - p||_\infty$ is the maximum error between $f$ and $p$ over $[-1,1]$. It is equal to $-e(x_1)$. So the error function $e(x)$ must achieve a local extremum at $x = x_1$ \\

In other words, $e'(x_1) = 0$. We can use this fact to determine $x_1$.
\begin{align*}
    e(x) &= f(x) - p(x) = e^x - a - bx \\
    e'(x) &= e^x - b \\
    0 &= e^{x_1} - b \\
    x_1 &= \ln{b} \approx 0.16144
\end{align*}

Now using $e(x_1) = -e(x_2)$
\begin{equation*}
    e^{x_1} - a - bx_1 = -e + a + b
\end{equation*}

Since $e^{x_1} = b$ (by our previous calculation), we see that
\begin{align*}
    -a - bx_1 = -e +a \\
    a = \frac{e - bx_1}{2} \approx 1.2643
\end{align*}

so the best line is \fbox{$p_x \approx 1.2643 + 1.1752x$} \\

Just what is the maximum error? \\

Well it occurs at $x_0, x_1, x_2$, so computing we see 

\begin{equation*}
    \boxed{||f - p||_\infty \approx 0.2788}
\end{equation*}

The previous example shows how one might construct the best uniform approximating polynomial of a certain degree. But what if we're not sure what degree $n$ we're interested in? We might want to approximately compute the number $d(f, \Pi_n)$, in order to find a good approximation. After all, if $d(f, \Pi_n)$ is too big, then we know that even the best approximation polynomial of degree $n$ isn't very good. In that case, we might wish to look at higher degree polynomials. \\

\underline{Can we estimate $d(f, \Pi_n)$?} \\

Well, sort of. We \underline{can} compute a \underline{lower bound}. That's useful, in that it helps us eliminate degrees $n$ for which $d(f, \Pi_n)$ is too big. \\

Here is the procedure. \\

I'll only show you how to compute $d(f, \Pi_1)$, that is a lower bound for $d(f, \Pi_1)$. But the basic method generalizes to arbitrary $n$ \\

[ Recall what $d(f, \Pi_1)$ means. It is the smallest $L_\infty$ error possible if one approximates $f(x)$ by linear polynomials.] \\

We start with a function $f: [a,b] \rightarrow \mathbb{R}$.

Recall the notation of divided differences, e.g. $f[x_0, x_1, x_2]$. \\

Suppose $p \in \Pi_1$ \\

Let $x_0, x_1, x_2$ be any three points in the interval $[a,b]$. Since $p$ is linear,
\begin{equation*}
    p[x_0, x_1, x_2] = 0
\end{equation*}

And we can write
\begin{equation*}
    f[x_0, x_1, x_2] = \frac{f(x_0)}{(x_0 - x_1)(x_0 - x_2)} + \frac{f(x_1)}{(x_1 - x_0)(x_1 - x_2)} + \frac{f(x_2)}{(x_0 - x_2)(x_1 - x_2)}
\end{equation*}
\begin{align*}
    f[x_0, x_1, x_2] &= f[x_0, x_1, x_2] - p[x_0, x_1, x_2] \\
    &= (f -p)[x_0, x_1, x_2] \\
    &= \frac{f(x_0) - p(x_0)}{(x_0 - x_1)(x_0 - x_2)} + \frac{f(x_1) - p(x_1)}{(x_1 - x_0)(x_1 - x_2)} + \frac{f(x_2) - p(x_2)}{(x_0 - x_2)(x_1 - x_2)}
\end{align*}
where $w(x) = (x - x_0)(x-x_1)(x-x_2)$ \\

(So $w'(x) = (x - x_1)(x - x_2) + (x - x_0)(x - x_2) + (x - x_0)(x - x_1)$) \\

\begin{align*}
    |f[x_0, x_1, x_2]| \leq ||f - p||_\infty \left( \frac{1}{|w'(x_0)|} + \frac{1}{|w'(x_1)|} + \frac{1}{|w'(x_2)|} \right) \\
    ||f - p||_\infty \geq \frac{|f[x_0, x_1, x_2]|}{\frac{1}{|w'(x_0)|} + \frac{1}{|w'(x_1)|} + \frac{1}{|w'(x_2)|}}
\end{align*}

First consider the left hand side of this inequality. The polynomial $p$ is arbitrary (within $\Pi_1$). None of its coefficients appear in the right hand side. We can therefore conclude that

\begin{equation*}
    d(f, \Pi_1) = \min_{p \in \Pi_1}{||f - p||_\infty} \geq \frac{|f[x_0, x_1, x_2]|}{\frac{1}{|w'(x_0)|} + \frac{1}{|w'(x_1)|} + \frac{1}{|w'(x_2)|}}
\end{equation*}

Now look at the right hand side of this inequality. It depends only of $f$ and on $x_0, x_1, x_2$. Furthermore, the points $x_0, x_1, x_2$ are arbitrary. So, we can conclude that
\begin{equation*}
    d(f, \Pi_1) \geq \max_{x_0, x_1, x_2} {\frac{|f[x_0, x_1, x_2]|}{\frac{1}{|w'(x_0)|} + \frac{1}{|w'(x_1)|} + \frac{1}{|w'(x_2)|}}}
\end{equation*}

This is the lower bound we are seeking. Of course, determining the max is usually too difficult. Instead it is often easier simply to pick three points $x_0, x_1, x_2$ and compute the resulting quotient. This, as we saw, is certainly also a lower bound for $d(f, \Pi_1)$, though perhaps not the highest possible.

\underline{Ex.} Let's try this out on our example $f(x) = e^x$ on $[-1,1]$. Taking $x_9 = -1, x_1 = 0, x_2 = 1$, we get
\begin{align*}
    f[x_0, x_1, x_2] = \frac{1}{2} f(-1) - f(0) + \frac{1}{2} f(1) \\
    \frac{|f[x_0, x_1, x_2]|}{\frac{1}{|w'(x_0)|} + \frac{1}{|w'(x_1)|} + \frac{1}{|w'(x_2)|}} = \frac{1}{2} + 1 + \frac{1}{2} = 2 \\
    d(f, \Pi_1) \geq \frac{f(-1) - 2f(0) + f(1)}{4}
\end{align*}

Notice this is true for all $f$ whose interval of definition includes $[-1,1]$. Now let's plug in for $f(x) = e^x$
\begin{equation*}
    d(e^x, \Pi_1) \geq \frac{e^{-1} - 2e^0 + e^1}{4} \approx 0.2715
\end{equation*}

\underline{Comments}
\begin{itemize}
    \item Notice that this lower bound is pretty close to the best error possible, namely $0.2788$
    \item This lower bound tells us that we can't expect to approximate $e^x$ with a line that gives decimal place accuracy. If we want better than about $\pm 0.3$ accuracy, we need higher-order approximations.
\end{itemize}

\underline{Ex.} Consider the function $f(x) = x^{n+1}$ on the interval $[-1,1]$. Suppose we wanted to approximate this degree $n+1$ polynomial with polynomials of degree at most $n$. \\

\underline{Comment}: 
\begin{itemize}
    \item There is no real reason we would want to do this, but it will turn out the best uniform approximation can actually be obtained by interpolate!
    \item \underline{Why is that nice?} Well, computing the best uniform approximation to a function $f$ by polynomials in $\Pi_n$ can be quite difficult. Interpolation is easy. So if we can find interpolation points whose interpolating polynomial is ``almost best'' (in the uniform sense), then we've made life a lot easier.
\end{itemize}

Recall the Chebyshev polynomials of degree $k$:
\begin{equation*}
    T_k(\cos{\theta}) = \cos{k\theta}
\end{equation*}
and in general we have the recurrence
\begin{align*}
    &T_{k+1}(x) = 2x T_k(x) - T_{k-1}(x)~~,~~k = 1,2, .. \\
    \text{with }& T_0(x) = 1~\text{and}~T_1(x) = x
\end{align*}

\underline{Observe}:
\begin{enumerate}
    \item $|T_k(x)| \leq 1$ for all $x \in [-1,1]$
    \item The leading coefficient of $T_k$ is $2^{k-1}~~,~~k = 1,2, ..$
    \item $T_k(x) = \pm 1$ alternatingly at the $k+1$ points \\
    $$x_j = \cos{\frac{k - i}{k}} \pi~~,~~j = 0, ... , k$$
    [Why? Becuase $T_k(x_j) = T_k(\cos{\frac{k - j}{k}} \pi) = \cos{((k-j)\pi)} = (-1)^{k-j}$]
\end{enumerate}

It follows that $x^{n+1} - 2^{-n} T_{n+1}(x)$ is a polynomial of degree at most $n$. Call it $p_n(x)$. \\

Observe that $x^{n+1} - p_n(x) = 2^{-n} T_{n+1}(x)$ \\

The right hand side satisfies the alternating-error-condition (i.e. alternating error at $n+2$ points, equal to the maximum possible error between $f$ and $p_n$) of our theorem. \\

So, $p_n$ must be the best uniform approximation! \\

\underline{Conclusions:}
\begin{enumerate}
    \item $d(x^{n+1}, \Pi_n) = \frac{1}{2^n}$
    \item Since $T_{n+1}(x)$ is zero at the $n+1$ Chebyshev points $\zeta_{k, n+1} = \cos{ \frac{2k+1}{2n+2} \pi} ~~,~~k = 0,1, ..$, we see that $p_n(x)$ is the polynomial that interpolates $x^{n+1}$ at $\zeta_{0, n+1}, ... \zeta_{n, n+1}$
\end{enumerate}

So amazingly, for the specific function $f(x) = x^{n+1}$, we can obtain the best uniform approximation by polynomials of degree at most $n$ simply by interpolating! \\

As we mentioned back when we were studying interpolation, it turns out that for arbitrary $f$, interpolating at the Chebyshev points or the expanded Chebyshev points comes very close to producing the best uniform approximation. Amazing. \\

See CdB, pp. 242-244 for further details.

\newpage
\section{Data Fitting}

Now imaging we have taken measurements of some unknown function $f$ at points $x_1, ... , x_n$. We would like to ``reconstruct'' this function as well as we can.\\

(Some applications should pop into your head: surface reconstructions, signal recovery, and even learning, say if observed robot dynamics.)\\

Now let's refer to our $n$ measurements as $f_i~,~i = 1, ... ,n$. If the measurements were perfect, then for each $i$, we would have $f_i = f(x_i)$. Generally, however, the measurements have been corrupted with noise or other errors, so that
\begin{equation*}
    f_i = f(x_i) + \epsilon_i~,~\text{where } \epsilon_i \text{ is some unknown error}
\end{equation*}

Nonetheless, we would like to try to recover $f(x)$. Were it not for the measurement errors, we might consider using interpolation. But there's no point in forcing our interpolating polynomial to pass exactly through points $(x_i, f_i)$ that are errorful. Indeed, the polynomial may just wiggle around a lot to pass through all the points $(x_i, f_i)$ and thus have an order much higher than $f(x)$ really has. \\

\underline{Ex.} Let's look again at our old friend, the function $f(x) = (x-1)^2$. Suppose we have taken measurements at 7 points, evenly spaced over the interval $[-1,2]$. The following table lists the points $x_i$, the true values $f(x_i)$, and our (errorful) measurements.

\begin{table}[H]
\centering
\begin{tabular}{ccc}
\hline
$x_i$ & $f(x_i)$ & $f_i$ \\ \hline
-1   & 4       & 4.1  \\
-0.5 & 2.25    & 2.3  \\
0    & 1       & 1.05 \\
0.5  & 0.25    & 0.20 \\
1    & 0       & 0.05 \\
1.5  & 0.25    & 0.26 \\
2    & 1       & 0.90 \\ \hline
\end{tabular}
\end{table}

The function $f(x)$ is a quadratic. If we interpolated through the points $\{(x_i, f_i)\}$ we would get a 6th order polynomial. No doubt that polynomial would contain some twists and turns that $f(x)$ does not (in other words, even if we get lucky and the interpolating polynomial matches $f(x)$ well, its derivatives may not). \\

\underline{Note:} Of course, there are many functions $\{g(x)\}$ that could produce the measured values $f_i$ above. How could we possibly hope to recover the correct function (in this case $(x-1)^2$) without further information we can't expect to recover the correct one. Suppose, however, that we know the underlying function is quadratic. (Or maybe we decide that we will only look at quadratic functions.)\\

Then we could pick as \underline{basis functions} the functions:
\begin{equation*}
    1~,~x~,~x^2
\end{equation*}
and seek to find coefficients $a,b,c$ such that
\begin{equation*}
    f(x) = a(1) + b(x) + c(x^2)
\end{equation*}

(Note: other basis functions are possible and indeed sometimes desirable. The key is that they are independent and span the vector space of quadratic functions. Later we'll also look at orthogonal bases.) \\

\underline{If} our measurements $f_i$ were perfect, then we would know
\begin{equation}
    \label{eq:07_01}
    f_i = a + b x_i + c x_i^2~~,~~ i = 1, ... , n
\end{equation}

This $n \times 3$ system of equations in the variables $a, b, c$ has a unique solution if the $f_i$ are perfect. \\

If the $f_i$ are imperfect, then the system (Equation \ref{eq:07_01}) is generally over-constrained. But we could still obtain a least-squares solution, using for example SVD! \\

That's exactly the approach we will take. We will now look at the general formulation of this simple idea. \\

Suppose we are given $n$ measurements $(x_i, f_i)$. We would like to reconstruct the function $f(x)$ \\

In practice we have some parameterized family of functions 
\begin{equation*}
    F(x) = F(x; c_1, ... , c_k)
\end{equation*}

We then choose the parameters $c_1, ... , c_k$ based on observations $\{(x_i, f_i)\}$ in such a way that
\begin{equation*}
    F(x)~\text{is ``close to''}~f(x)
\end{equation*}

Often, for mathematical simplicity we write $F(x)$ as a linear combination of some $k$ basis functions:
\begin{equation*}
    F(x) = c_1 \phi_1(x) + ... + c_k \phi_k(x)
\end{equation*}

This is what we did in the example before. Again the objective is to choose the $\{c_i\}$ well.\\

\underline{Comment:} Why do we do things this way? \\

The point is that $k$ will generally be much smaller than $n$. So rather than retain all $n$ pieces of data (as with interpolation), most of which really contains little information (due to error), we try to extract the important or useful information. That information is encoded in the $k$ parameters $\{c_i\}$.\\

\underline{Next question: How does one choose the $\{c_j\}$?} \\

Ideally we would like to minimize the difference between $f(x)$ and $F(x)$. Unfortunately, we don't know $f(x)$, so instead we simply minimize the difference between $f$ and $F$ at the data points $x_1, ... , x_n$. \\

Even then, we could use a variety of norms by which to measure the error:
\begin{align*}
    \text{\underline{$\infty$-norm:}}~~&~~||f - F||_\infty = \max_{1 \leq i \leq n}{|f_i - F(x_i)|} \\
    \text{\underline{1-norm:}}~~&~~||f - F||_1 = \sum_{i = 1}^{n}{|f_i - F(x_i)|} \\
    \text{\underline{p-norm:}}~~&~~||f - F||_p = \sqrt[p]{\sum_{i = 1}^{n}{|f_i - F(x_i)|^p}}
\end{align*}

Recall that the parameters $\{c_j\}$ are hidden in this notation. In face, $F(x_i) = F(x_i; c_1, ... , c_k)$. For each norm, the goal would be to choose the $\{c_j\}$ so as the make the error a minimum. \\

This minimization process tends to lead to \underline{non-linear} equations in $c_1, ... , c_k$, even when $F(x)$ has the simple linear form $F(x) = c_1 \phi_1(x) + ... + c_k \phi_k(x)$. However, as we shall see, in the case of a 2-norm, the error minimization leads to linear equations that determine $c_1, ... , c_k$ (assuming $F(x)$ itself has a simple linear form). For this practical reason, 2-norms, that is \underline{least squares} are so popular. In other words, we want to choose the $\{c_j\}$ to minimize the quantity
\begin{equation*}
    ||f - F||_2 = \sqrt{\sum_{i = 1}^{n}{|f_i - F(x_i; c_1, ... , c_k)|^2}}
\end{equation*}

\subsection{Least Squares}
It is enough to choose $c = (c_1, ... , c_k)$ to minimize this function
\begin{equation*}
    E(c) = \sum_{i=1}^{n} (f_i - F(x_i; c))^2
\end{equation*}

At the point $c \in \mathbb{R}^k$ where $E(c)$ attains a minimum, it must not be the case that each of the partials vanishes, that is, 
\begin{equation*}
    \frac{\partial}{\partial c_j} E(c) = 0~~,~~j = 1, ... , k
\end{equation*}

Computing, we see that
\begin{equation*}
    \frac{\partial}{\partial c_j} E(c) = -2 \sum_{i=1}^{n} (f_i - F(x_i; c)) \frac{\partial}{\partial c_j} F(x_i; c)
\end{equation*}

Given that $F(x; c) = \sum_{j=1}^{n} c_j \phi_j (x)$, we have
\begin{equation*}
    \frac{\partial}{\partial c_j} F(x_i; c) = \phi_j(x_i)~~~\text{(a number)}
\end{equation*}

Therefore, $\frac{\partial}{\partial c_j} E(c) = -2 \sum_{i=1}^{n} (f_i - F(x_i; c)) \phi_j(x_i)$. \\

When $E(c)$ is a minimum, we therefore have the system:
\begin{equation*}
    \sum_{i=1}^{n} (f_i - F(x_i; c)) \phi_j(x_i) = 0~~~j = 1, ... k
\end{equation*}

This system of equations is called the system of \underline{normal equations}.\\

Why this name? \\

Well, consider the error vector
\begin{equation*}
    \vec{e} = \begin{pmatrix} f_1 - F(x_1; c) \\ . \\ . \\ . \\ f_n - F(x_n; c) \end{pmatrix}
\end{equation*}

And consider the k vectors
\begin{equation*}
    \vec{\phi}_j = \begin{pmatrix} \phi_j(x_1) \\ . \\ . \\ . \\ \phi_j(x_n) \end{pmatrix}~~~j = 1, ... k
\end{equation*}

$\vec{e}$ measures the error of the $n$ data points. $\vec{\phi}_j$ encodes the value of the basis function $\phi_j$ at the $n$ data points.
}

\textbf{Roshni - Pages 21 - 48 of notes remaining}
    \chapter{Integration of Ordinary Differential Equations}

\subsection*{Applications}
\begin{itemize}
    \item Understanding the dynamics of some physical system. Classic examples include aerodynamics, fluid dynamics, predator-prey, heat diffusion, mechanical oscillations (drums, beams), etc. Other examples include robots bouncing, bending (see Raibert, Canon, Koditschek, Shuttle arm)
    \item Developing control algorithms and strategies
    \item Solving optimization problems
    \item Understanding stochastic systems
\end{itemize}

\section{First-order initial value problems}

\subsection{Taylor's algorithm}

\subsection{Error Estimates}

\subsubsection{Local Discretization Error}

\subsubsection{Full Discretization Error}

\subsubsection{Numerical Roundoff Error}

\subsection{Local Error}

\subsection{Full Discretization Error}

\subsection{Runge-Kutta Methods}

\subsubsection{Runge-Kutta Method of Order 2}

\subsubsection{Runge-Kutta Method of Order 4}

\subsection{Multi-Step Formulas (Adams-Bashforth}

\subsection{Comparison of Runge-Kutta and Adams-Bashforth}

\subsection{Omitted Topics}

\section{Boundary Value Problems}

\subsection{Finite-Difference Method}

\subsection{Central Difference Derivations}

\subsection{Relaxation}

\subsection{Successive Overrelaxation}

\subsection{Omitted Topics}
    \chapter{Optimization}

Worked on by Alex.

This is another vast and deep subject, and we will only take a brief peak.
We have of course already seen some forms of optimization, in the form of error minimization.
No doubt you are familiar as well with the optimization of linear systems.
In this set of notes we will focus on optimizing arbitrary non-linear functions.
And in the next section of notes we will consider optimization of functionals, that is, functions of functions.

\section{Applications (Let's focus on robotics-related optimization)}

\begin{itemize}
  \item Minimize energy/power consumption.
  \item Minimize torque, null space forces.
  \item Path planning via potential fields.
  \item Computation of stable resting configurations.
  \item Maximize sensing information. (e.g. where to probe next?)
  \item Optimize execution time (Actually, this leads to Markov Decision Theory, Dynamic Programming, and Calculus of Variations. We will discuss C of V in detail later.)
\end{itemize}

We will now become increasingly more analytical and less numerical.
You should also put on your geometric thinking caps, since much of the analysis is best understood in terms of geometry.

    \chapter{Calculus of Variations}
Worked on by Ratnesh

The calculus of variations arose in tight complicity with the development of mechanics. 
Much of the axiomatic grounding of physics in general and mechanics in particular consists of variational principles. 
Indeed, the rudiments of quantum theory may be derived from Newton's laws and variational principles. 
We will in these notes trace some of the history of the calculus of variations. 
One caution: while much of the treatment will seem mathematical, it is in fact very handwavy. 
A true and correct treatment would require us to pull out such differential geometry tools as covariant derivatives, affine connections, and n-forms. 

Applications:
\begin{itemize}
    \item Path optimizational
       \begin{itemize}
            \item Optimal control (minimum cost trajectories)
        \end{itemize}
    \item Engineering
        \begin{itemize}
            \item Vibrating membranes
            \item Theory of elasticity
            \item Electrostatistics
        \end{itemize}
    \item Machine Vision
        \begin{itemize}
            \item Surface reconstruction
            \item Image flow (Motion and Structure from Optical Flow)
            \item Edge detection
        \end{itemize}    
\end{itemize}

\section{Introduction}
We have seen the basic principle ``To minimize $P$ is to solve $P' = 0$"

So far, we have only looked at finite-dimensional problems, that is, minimization of some function $f: \mathbb{R}^n \mapsto \mathbb{R}$. 
In such a problem, we seek the value of $n$ numbers that minimize $f$. 

What about infinite-dimensional problems, that is, problems in which $P$ depends on an infinity of numbers?

In particular, what about functionals (functions of functions)?

\bigbreak

\textbf{Example}:\\
Suppose we connect two points in the plane, $(x_0, y_0)$ and $(x_1, y_1)$ by a rectifiable curve of the form $y = y(x)$. 



\begin{figure}
    \centering
    \includegraphics[width=\textwidth ]{figures/placeholder.png}
    \caption{bla.}
    \label{fig:}
\end{figure}

    \chapter{Markov Chains}

Worked on by Akash

Let's first recall the definition of independent trials.
\begin{definition}[Independent trials]
A set of possible outcomes $E_1,E_2,...$ is given. With each outcome $E_k$ there is associated a probability $p_k$. The probability of a sampled sequence is defined as
$$ P\{(E_{j_0},E_{j_1},...,E_{j_k})\} = p_{j_0} p_{j_1} \hdots p_{j_k}$$
\end{definition}

\medskip
\noindent In the theory of Markov Chains the outcome of any trial depends on the outcome of the directly preceding trial only. 

\medskip
\noindent \textbf{Conditional probability}
\begin{itemize}
    \item $p_{jk}$: given that $E_j$ has occurred at some trial the probability of $E_k$ at the next trial.
    \item $a_k$ : probability of $E_k$ at the initial trial 
\end{itemize}
For instance here are the prob. of some sample sequences.
\begin{align*}
    &P\{(E_j,E_k)\} = a_j p_{jk}\\
    &P\{(E_j,E_k,E_r) \} = a_j p_{jk} p_{kr}\\
    \textrm{(and generally) }& P\{(E_{j_0},E_{j_1},...,E_{j_n})\} = a_{j_0} p_{j_0 j_1} p_{j_1 j_2} \hdots p_{j_{n-1} j_n}
\end{align*}

\section{Random Walk}
A random walk is the process by which randomly-moving objects wander away from where they started.
The simplest random walk to understand is a 1-dimensional random walk. Suppose that 
Given a set of events: 
$$ \{\dots -3, -2, -1, 0, 1, 2, 3, \dots\}$$
we have $P_{j,k} = 0$ if $ | j - k | > 1$. For a symmetric random walk we might accordingly have: $ P_{i, j} = 0$\\
$ P_{j, k} = \frac{1}{2}$ if $|j - k| = 1$
Graphically we would draw arrows with labelled probabilities. It is often a good way to think of a random walk as a markov chain as follows: 

\akash{Small figure here}

\begin{definition}
A sequence of trials with possible outcomes $E_1, E_2, \dots$ is called a Markov chain if the probabilities of sample sequeces are defined by: 
$ P(E_{j_0}, E_{j_1}, \dots , E_{j_n}) = a_{j_0}p_{j_0, j_1} \dots p_{j_{n-1} j_n}$ in terms of a probability distribution $\{a_k\}$ for $E_k$ at the initial (or zero-th) trial and fixed conditional probabilities $p_{j, k}$ of $E_k$ given that $E_j$ has occured in the preceding trial.
\end{definition}
where: $E_k$ = states of the system, $a_k$ = the probabilities of $E_k$ at the initial trial $P_{j, k}$ the probabilities of the transition from $E_j$ to $E_k$

Furthermore, the matrix (finite or infinite) of transition probabilites: 
\begin{align*}
    \mathbf{P} = \begin{bmatrix} 
                    P_{11} & P_{12} & P_{13} & \dots \\
                    P_{21} & P_{22} & P_{23} & \dots \\
                    P_{31} & P_{32} & P_{33} & \dots \\
                    \vdots & \vdots & \vdots & \ddots \\
                \end{bmatrix}
\end{align*}
is a square matrix with non-negative elements and unit row sums. Such a matrix is called a \underline{Stochastic Matrix}.

Any stochastic matrix with initial distribution $\{a_k\}$ completely defines a markov chain with states $E_1, E_2, \dots$

\begin{example}
There are only two possible states $E_1$ and $E_2$. The matrix of transition probabilities is: 
\begin{align}
    \mathbf{P} = \begin{bmatrix} 
                 1 - p & p \\
                 \alpha & 1 - \alpha \\
                 \end{bmatrix}
\end{align}
or in graphical form: 
\akash{Add short figure}
\end{example}

\begin{remark}
This chain could be realized by the following notion of a particle moving along the x-axis. 
It's absolute speed remains constant but the direction of the motion can be reversed. 
The system is in state $E_1$, if the particle moves in the positive direction and in state $E_2$ if the motion is to the left.
Then $p$ is the probability of a reversal when the particle moves to the right and $\alpha$ is the probability of the reversal when the particle moves to the left.
\end{remark}

\begin{example}[Random walks with Absorbing barriers]

\akash{Add short figure}
\begin{align}
    \mathbf{P} = \begin{bmatrix}
    1 & 0 & 0 & 0 \dots 0 & 0 & 0 \\
    q & 0 & p & 0 \dots 0 & 0 & 0 \\
    \vdots & \vdots & \vdots & \dots & \vdots & \vdots\\
    0 & 0 & 0 & 0 \dots q & 0 & p \\
    0 & 0 & 0 & 0 \dots 0 & 0 & 1
    \end{bmatrix}
\end{align}
\begin{itemize}
\item Each of the "interior" states $E_1, \dots, E_{n-1}$ admit transitions only to its left neighbor (with probability $P$) and its right neighbor (with probability $Q$)
\item No transition is possible form $E_0$ or $E_n$ to any other statye. Once $E_0$ or $E_n$ is reached, the system stays there forever
\end{itemize}
\end{example}

\begin{example}[Reflecting barriers]
\akash{Add another figure here}
\begin{align}
    \mathbf{P} = \begin{bmatrix} 
    q & p & 0 & 0 & \dots & 0 & 0 & 0 \\
    q & 0 & p & 0 & \dots & 0 & 0 & 0 \\
    \vdots & \vdots & \vdots & \dots & \vdots & \vdots\\
    0 & 0 & 0 & 0 & \dots & q & 0 & p \\
    0 & 0 & 0 & 0 & \dots & 0 &q & p \\
    \end{bmatrix}
\end{align}
\end{example}

\begin{example}[Cyclical random walks]

\begin{align}
 \mathbf{P} = \begin{bmatrix} 
    0 & p & 0 & 0 & \dots & 0 & 0 & q \\
    q & 0 & p & 0 & \dots & 0 & 0 & 0 \\
    0 & q & 0 & p & \dots & 0 & 0 & 0 \\
    \vdots & \vdots & \vdots & \dots & \vdots & \vdots\\
    0 & 0 & 0 & 0 & \dots & q & 0 & p \\
    p & 0 & 0 & 0 & \dots & 0 & 0 & q \\
    \end{bmatrix}   
\end{align}
More generally we may permit the transitions between any two states. Let $q_0, q_1, \dots, q_{n-1}$ be respectively the probabilities of staying fixed or moving $1, 2, \dots, n-1$ units to the right where $k$ units to the right is same as $n-k$ units to the left. The P is as shown below:
\begin{align}
    \mathbf{P} = \begin{bmatrix}
    q_0 & q_1 & q_2 & \dots & q_{n-2} & q_{n-1} \\
    q_{n-1} & q_0 & q_1 & \dots & q_{n-3} & q_{n-2} \\
    \vdots & \vdots & \vdots & \dots & \vdots & \vdots \\
    q_1 & q_2 & q_3 & \dots & q_{n-1} & q_0
    \end{bmatrix} 
\end{align}
\end{example}

\section{Higher Transition Probabilities}

Denoted by $P_{jk}^l$ the probability of a transition form $E_j$ to $E_k$ in exact $l$ steps. This is the sum of the probabilities of all possible paths $E_j, E_{j-1}, \dots, E_{j-{l-1}}, E_k$, i.e., 
\begin{align}
    P_{jk}^l = \sum_{v \in (j_1, j_{l-1})} \mathbf{P} {(E_v)}
\end{align}

In particular $P_{jk}^1$ = $P_{jk}$ and $P_{jk}^2 = \sum_s P_{js} P_{sk}$. 
We define \[P_{jk}^{(j)} = \begin{cases} \mbox{1,} & \mbox{if } j = k \\ \mbox{0,} & \mbox{if $j \neq k$} \end{cases} \]
and recursively we have:
\begin{align}
    P_{jk}^{i+1} = \sum_s P_{js}P_{sk}^i
\end{align}
In fact, if we arrange $P_{jk}^l$ into a matrix denoted by $\mathbf{P}^l$, then $\mathbf{P}^l$ is just the chain product of l identical matrices $\mathbf{P}$.
\begin{example}
For independent trials it is clear without calculation that $\mathbf{P}^l = \mathbf{P} for all l$
\end{example}

    \chapter{Two-Dimensional Configuration Space}


    \input{13-convex-hull.tex}
    \chapter{Two-Dimensional Voronoi Diagram}
    \chapter{Differential Geometry: Frenet Frames and Surface Curvature}

\begin{remark}
  Material largely adopted from ``Elementary Differential Geometry''
  by B. O'Neill
\end{remark}

Differential geometry studies the motions possible in a space. Some
key concepts:

\begin{definition}[Tangent Vector $v_p$]
  A vector anchored at a particular
point $p$. Set of all possible $v_p$ for a given $p$ is called the tangent
space $T_p$ at $p$.
\end{definition}

\anirudh{Small figure here}

\begin{definition}[Tangent Bundle]
  A space along with all its tangent vectors.
\end{definition}


\begin{example}
  If $\Real^n$ is the underlying space then we have another $\Real^n$
  at each point $p \in \Real^n$ consisting of all the tangent vectors
  anchored at $p$. So we get $\Real^n \times \Real^n$ over all, just
  like a state space. For $n = 2$:
  \anirudh{Example figure here}
\end{example}

\begin{example}
  The tangent bundle associated with a circle looks like
  $\Sphere^1 \times \Real^1$:
  \anirudh{Example figure here}
\end{example}

\textbf{Vector field} A function $M \rightarrow T(M)$ where $M$ is the
underlying space, often called a \textit{manifold} (e.g. $\Real^n$,
$\Sphere^n$, even matrix groups e.g. SO(n)), and $T(M)$ is the tangent
bundle. For a point $p$ it defines a mapping $p \rightarrow v_p \in
T_p$ often denoted by $V(p)$ or $V_p$ (We generally require $V$ to be
smooth, meaning as many derivatives as we need.)

One classic question is whether a manifold has a continuously varying
vector field that is never zero. 

\end{document}

%%% Local Variables:
%%% mode: latex
%%% TeX-master: t
%%% End:
