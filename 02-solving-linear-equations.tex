\chapter{Solving Linear Equations}

We have discussed general methods for solving arbitrary equations. For the most part, we restricted our attention to the one-dimensional case. We also looked at a special class of equations, namely those given by polynomials. Another special class is given by linear equations. Of course, the one-dimensional case is absolutely trivial, so one naturally considers systems of equations. 

\noindent I assume that you are familiar with the basic results of linear algebra, such as the diagonalization theorems and Gaussian Elimination. This section provides a very quick review, then discusses Singular Value Decomposition (SVD).

\subsection*{Applications}

Applications for linear equations are numerous. These include, of course, Linear Programming problems. Other are the local solution of non-linear differential equations via linearization, and the determination of minima in least-squares problems. 

\newpage

\section{A quick review of important definitions and facts}

\begin{enumerate}
\item Given a linear function $f: \mathbb{R}^n \rightarrow \mathbb{R}^m$ and a pair of bases,
    \begin{align*}
        \mathcal{B}_1 &= \{e_1, \dots, e_n\}\text{ for }\mathbb{R}^n \\
        \mathcal{B}_2 &= \{d_1, \dots, d_m\}\text{ for }\mathbb{R}^m
    \end{align*}
    we can represent $f$ by an $m\times n$ matrix $A$. \\

    Notice that the matrix $A$ depends on the choice of bases $\mathcal{B}_1 \times \mathcal{B}_2$.
    Often we just let $\mathcal{B}_1 \times \mathcal{B}_2$ be the "obvious" bases for $\mathbb{R}^n \times \mathbb{R}^m$. 
    
    \textbf{Moral:} A matrix is a particular coordinate representation of the linear function $f$.
    
\item Given an $m \times n$ matrix $A$, we make the following definitions:
    \begin{itemize}
        \item Column space --- linear combinations of the columns of $A$
        \item Row space --- linear combinations of the rows of $A$
    \end{itemize}
    
    If we think of $A$ as defining the linear mapping 
    \begin{align*}
        A: \mathbb{R}^n &\rightarrow \mathbb{R}^m 
        \\
        \begin{bmatrix} x_1 \\ \vdots \\ x_n \end{bmatrix}
            &\rightarrow
            \begin{bmatrix} y_1 \\ \vdots \\ y_m \end{bmatrix}
            =
            A
            \begin{bmatrix} x_1 \\ \vdots \\ x_n \end{bmatrix}
    \end{align*}
    then the column space of $A$ is a vector subspace of $\mathbb{R}^m$, consisting of all points in $\mathbb{R}^m$ that are image vectors under $A$. Note that the row space of $A$ is just the column space of $A^T$, the transpose of $A$. \\
    
    And we define
    \begin{itemize}
        \item Null space --- set of vectors $x$ in $\mathbb{R}_n$ such that $Ax=0$.
    \end{itemize}

    The following relationships are useful to remember:
    \begin{itemize}
        \item $\dim(\text{row space}) = \dim(\text{column space})$, either of which is called the \underline{rank} of $A$,
        \item The row space and null space are complementary (perpendicular) subspaces of $\mathbb{R}^n$. In other words, 
        \begin{align*}
            \mathbb{R}^n &= \text{row space}\oplus\text{null space} \\
            n &= \dim(\text{row space}) + \dim(\text{null space})
        \end{align*}
    \end{itemize}
    In picture form:
    \begin{figure}[H]
        \caption{The figure shows a point $x = x_r \oplus x_n$ being mapped through a matrix $A$ such that $y=Ax$. Here $Ax = Ax_r + Ax_n = Ax_r + 0 = Ax_r$.}
        \label{fig:linear-1}
        \centering
        \begin{tikzpicture}[vect/.style={->,shorten >=3pt,>=latex'}]
    \begin{scope}[rotate=-30]
        \tkzDefPoint(0,0){O}
        \tkzDefPoint(3,0){A}
        \tkzDefPoint(0,3){B}
        \tkzDefPoint(2,1){x}
        \tkzDefPoint(2,0){x_n}
        \tkzDefPoint(0,1){x_r}
        \tkzDrawSegments[vect](O,B O,A)
        \tkzDrawSegments[color=gray, style=dashed](x_r,x x_n,x)
    \end{scope}
    \tkzGetPointCoord(x_n){c}
    \begin{scope}[shift={(5+\cx,\cy)},rotate=30]
    \tkzDefPoint(0,0){O'}
    \tkzDefPoint(3,0){A'}
    \tkzDefPoint(0,3){B'}
    \tkzDefPoint(0,2){y}
    \tkzDrawSegments[vect](O',B' O',A')
    \end{scope}
    \tkzDrawSegments[vect](x,y x_r,y x_n,O')
    \tkzLabelPoints(x,y)
    \tkzLabelPoints[below](x_n)
    \tkzLabelPoints[left](x_r)
    \tkzDrawPoints(O',x_r,x_n,x,y)
    \tkzLabelSegment[sloped](x_n,O'){$x_n\mapsto A x_n = 0$}
    \tkzLabelSegment[sloped](x,y){$x\mapsto A x$}
    \tkzLabelSegment[sloped](x_r,y){$x_r\mapsto A x_r = Ax$}
    \tkzLabelPoint[right](A){$\text{null space of }A$}
    \tkzLabelPoint[right](B){$\text{row space of }A$}
    \tkzLabelPoint[right](A'){$\text{null space of }A^T$}
    \tkzLabelPoint[right](B'){$\text{column space of }A$}
\end{tikzpicture} 

    \end{figure}

\item Suppose $A$ is an $m\times n$ matrix, and consider the system of equations 
    \begin{align*}
        Ax = b.
    \end{align*}
    \begin{itemize}
        \item If $b$ is not an element of the column space of $A$, then we say that the system is \underline{inconsistent} (or \underline{overdetermined}). \\
        \item If $b$ is in the column space of $A$ and the null space of $A$ is non-trivial, then we say that the system is \underline{underdetermined}. In this case there is a whole family of solutions, given by the affine set
            $$x_0 + N$$
        where $x_0$ is any particular solution $Ax_0 = b$, and $N$ is the null space of $A$. \\
        \item If $A$ is a $n\times n$ square matrix we say that $A$ is \underline{singular} iff:
            \begin{description}
                \item \quad $\det(A) = 0$
                \item \quad $\text{rank}(A) < n$
                \item \quad the rows of $A$ are not linearly independent
                \item \quad the columns of $A$ are not linearly independent
                \item \quad the dimension of the null space of $A$ is non-zero
                \item \quad A is not invertible.
            \end{description}
    \end{itemize}
\end{enumerate}

\newpage
\section{A quick review of matrix decompositions}

We will look at SVD in more detail.

\subsection{Factorizations based on elimination}

Given an $m\times n$ matrix $A$ (with $m \geq n = \text{rank}(A)$), we can write $A$ in the form 
\begin{align*}
    PA &= LDU
\end{align*}
where 
\begin{description}
    \item \quad $P$ is an $m\times m$ permutation matrix that specifies row interchanges, 
    \item \quad $L$ is an $m\times m$ square low-triangular matrix with 1's on the diagonal, 
    \item \quad $U$ is an $m\times n$ upper-triangular matrix with 1's on the diagonal, and 
    \item \quad $D$ is an $m\times m$ square diagonal matrix.
\end{description}

[More generally: we may also need to perform column interchanges on $A$, in which case we get two permutation matrices, so $P_1 A P_2 = LDU$. Yet more generally: $PA=LU$ with diagional entries of $U$ no longer required to be $1$.]

\subsubsection*{Notes}
\begin{enumerate}
    \item The entries on the diagonal of $D$ are sometimes called ``pivots" (after the Gaussian Elimination algorithm). 
    \item The product of the pivots is equal to $\pm\det(A)$ (sign depends on $P$; negative if odd row interchanges), whenever $A$ is a square matrix.
    \item If $A$ is symmetric and $P=I$, then $U=L^T$.
    \item If $A$ is symmetric positive-definite, then $U=L^T$ and the diagonal entries of $D$ are strictly positive.
\end{enumerate}

\subsubsection*{Examples}
\begin{enumerate}[label=(\alph*)]
\item $
        \begin{bmatrix}
            1 & 0 \\
            1 & 1 \\
            0 & -1
        \end{bmatrix}
        =
        \begin{bmatrix}
            1 & 0 & 0 \\
            1 & 1 & 0 \\
            0 & -1 & 1
        \end{bmatrix}
        \underbrace{\begin{bmatrix}
            1 & 0 & 0 \\
            0 & 1 & 0 \\
            0 & 0 & 1
        \end{bmatrix}}_{\mathrlap{\text{Could use any value here. Use $0$ if we want $\text{rank}(D) = \text{rank}(A)$.}}}
        \begin{bmatrix}
            1 & 0 \\
            0 & 1 \\
            0 & 0
        \end{bmatrix}
    $
    
    [If $m > n$, as in this example, then the last $m-n$ rows of $U$ are zero.]
\item
        $
        \begin{bmatrix}
            1 & 1 & 0 \\
            2 & 1 & -1 \\
        \end{bmatrix}
        =
        \begin{bmatrix}
            1 & 0 \\
            2 & 1
        \end{bmatrix}
        \begin{bmatrix}
            1 & 0 \\
            0 & -1 
        \end{bmatrix}
        \begin{bmatrix}
            1 & 1 & 0 \\
            0 & 1 & 1
        \end{bmatrix}
    $
\end{enumerate}

\subsubsection*{Algorithm}

Gaussian elimination dierctly yields this decomposition.

\subsubsection*{Application}

As with most decompositions/factorizations, the hope is to simplify solving the system $Ax=b$. 

Suppose $A$ is square and non-singular. Then solving $Ax=b$ really means solving $LDUx = Pb$.

In turn this entails solving two simpler problems: 
\begin{enumerate}[label=(\roman*)]
    \item $Ly = Pb$ (solve for $y$)
    \item $Ux = D^{-1}y$ (solve for $x$)
\end{enumerate} 
Each of these problems can be solved easily using forward or back substitution ($D^{-1}$ is easy to compute since $D$ is diagonal with non-zero entries).

\subsection{Factorizations based on eigenvalues}

These are the standard factorizations one learns in a linear algebra course. Two important ones are:

\begin{enumerate}[label=(\roman*)]
\item If $A$ is a square $n\times n$ matrix with $n$ linearly independent eigenvectors, then
    $$A = S\Lambda S^{-1}$$
    where $\Lambda$ is a diagonal matrix whose entries are the eigenvalues of $A$, and $S$ is a matrix whose columns are the eigenvectors of $A$.
    
    This factorization is not always possible. (One case in which it is possible occurs when $A$ has $n$ distinct eigenvalues.)

\item One can always decompose $A$ in Jordan form, i.e.,
    $$A = MJM^{-1}$$
    \begin{align*}
        \text{where }J &= 
        \begin{bmatrix}
            J_1 & \dots & 0 \\
            \vdots & \ddots & \vdots \\
            0 & \dots & J_s
        \end{bmatrix}
        \text{is a block matrix, such that each block }\\
    J_i &= 
    \begin{bmatrix}
        \lambda_i^1 & \dots & 0 \\
        \vdots & \ddots & \vdots \\
        0 & \dots & \lambda_i^s
    \end{bmatrix}
    \text{ with $\lambda_i$ an eigenvalue of $A$.}
    \end{align*}
    
    Here $s$ is the number of independent eigenvectors of $A$. $M$ consists of eigenvectors and ``generalized'' eigenvectors.
    
    \subsubsection*{Side note}
    
    What is a ``generalized'' eigenvector?
    Well, suppose a Jordan block is of the form 
    \[J_i = \begin{bmatrix}4 & 1 \\ 0 & 4\end{bmatrix} \tag*{($\lambda_i = 4$)} \]
    Then $x_1 = \begin{bmatrix} 1 \\ 0 \end{bmatrix}$ is an eigenvector
    and $x_2 = \begin{bmatrix} 0 \\ 1 \end{bmatrix}$ is a generalized eigenvector
    
    since $J_i x_1 = \lambda_i x_1$ and 
    $J_i x_2 = \underbrace{\lambda_i x_2 + x_1}_{\mathclap{\text{``not quite'' an eigenvalue.}}}$
\end{enumerate}

\subsection{Factorizations based on $A^TA$}

Let's look at two such factorizations, QR and SVD.

\begin{enumerate}[label=(\roman*)]
    \item \textbf{QR}
        
    Suppose $A$ is an $m \times n$ matrix with independent columns. We can factor $A$ as:
    \[
        A = QR \tag*{($Q$ is $m \times n$,  $R$ is $n \times n$)}
    \]
    
    $Q$ has the same column space as $A$, but its columns are orthonormal vectors. In other words, $Q^TQ = I$ ($QQ^T$ may not be $I$, if $A$ is not square). 
    
    $R$ is invertible and upper triangular. 
    
    Here are two possible algorithms for computing this factorization:
    \begin{enumerate}
        \item Use Gram-Schmidt to orthogonalize the columns of $A$. The resulting orthonormal vectors constitute the columns of $Q$. The matrix $R$ is formed by keeping track of the Gram-Schmidt operations (specifically, $R$ expresses the columns of $A$ as linear combinations of the columns of $Q$). 
        \item First, form $A^TA$. This is a positive definite symmetric matrix. (It is positive definite because the columns of $A$ are independent. 
        
        Second, compute the $LDU$ factorization of $A^TA$. One finds that $A^TA = LDL^T$, with $L$ lower-triangular and $D$ diagonal strictly positive. 
        
        Finally, let $R = D^{\frac{1}{2}}L^T$ and $Q=AR^{-1}$. 
    \end{enumerate}
    
    Unfortunately, both of these straightforward algorithms have poor numerical stability. in practice one uses a different algorithm. See \S 11.3 of NRiC for a good algorithm.
    
    \subsubsection*{Applications}
    
    \begin{enumerate}[label=(\alph*)]
        \item ``The QR algorithm'': This is an iterative method that repeatedly produces $QR$ factorizations of matrices derived from $A$, building the eigenvalues of $A$ in the process.
        \item Supposed we wish to solve an overconstrained system $Ax = b$ in the least-squares sense. The least-squares solution $\bar{c}$ is given by $\bar{x}=(A^TA)^{-1}A^Tb$, assuming the columns of $A$ are indepedent. But $Q^TQ = I$, so $\bar{x} = R^{-1}Q^Tb$. 
        
        So, we can obtain $\bar{x}$ by computing $Q^Tb$, then using backsubstitution to solve $R\bar{x} = Q^Tb$. This is numerically more stable than solving the system $A^TA\bar{x} - A^Tb directly$.
    \end{enumerate}
    
    \item \textbf{SVD}
    
    Any $m \times n$ matrix can be factored as 
    \[
        A = U \Sigma V^T
    \]
    \begin{align*}
        \text{where } &U \text{ is an $m \times m$ orthogonal matrix whose columns are the eigenvectors of $AA^T$} \\
        &V \text{ is an $n \times n$ orthogonal matrix whose columns are the eigenvectors of $A^TA$} \\
        &\Sigma \text{ is a diagonal $n \times n$ matrix of the form} 
        % \[
        %     \Sigma = \begin{bmatrix}
        %         \sigma_1 & & 0 \\
        %          & \sigma_k & \\
        %         0 & & 0
        %     \end{bmatrix}
        % \]
    \end{align*}
    
\end{enumerate}



\section{SVD in more detail}

\begin{figure}
  \centering
  \label{fig:linear-2}
  \caption{SVD in more detail}
  \resizebox{\columnwidth}{!}{
    \begin{tikzpicture}[vect/.style={->,shorten >=3pt,>=latex'}]
\tkzDefPoints{5/0/A, 10/0/B, 15/0/C};
\tkzDefPoint(0,0){O};
\tkzDefPoints{-2/0/Xn, 2/0/Xp, 0/2/Yp, 0/-2/Yn};
\tkzDrawSegments[<->](Xn,Xp Yn,Yp);
\tkzDrawCircle[R](O,1 cm);
\tkzLabelPoint[above]([yshift=3cm]O){In $R_n$ coords.};
\begin{scope}[shift=(A)]
  \begin{scope}[rotate=15]
    \tkzDefPoint(0,0){O};
    \tkzDefPoints{-2/0/Xn, 2/0/Xp, 0/2/Yp, 0/-2/Yn};
    \tkzDrawSegments[<->](Xn,Xp Yn,Yp);
    \tkzDrawCircle[R](O,1 cm);
  \end{scope}
\tkzLabelPoint[above]([yshift=3cm]O){In ``$V$" coords.};
\end{scope}
\begin{scope}[shift=(B)]
  \begin{scope}[rotate=-15]
    \tkzDefPoint(0,0){O};
    \draw (O) ellipse (1.5cm and 1cm);
    \tkzDefPoints{-2/0/Xn, 2/0/Xp, 0/2/Yp, 0/-2/Yn};
    \tkzDefPoints{2/0/S1, 0/1/S2};
    \tkzDrawSegments[<->](Xn,Xp Yn,Yp);
    \tkzLabelSegment[below](O,S1) {$\sigma_1$}
    \tkzLabelSegment[right](O,S2) {$\sigma_2$}
  \end{scope}
  \tkzLabelPoint[above]([yshift=3cm]O){In ``$U$" coords.};
\end{scope}
\begin{scope}[shift=(C)]
  \tkzDefPoint(0,0){O};
  \tkzDefPoints{-2/0/Xn, 2/0/Xp, 0/2/Yp, 0/-2/Yn};
  \tkzDrawSegments[<->](Xn,Xp Yn,Yp);
  \begin{scope}[rotate=-15]
    \draw (O) ellipse (1.5cm and 1cm);
  \end{scope}
  \tkzLabelPoint[above]([yshift=3cm]O){In $R_m$ coords.};
\end{scope}
\tkzDefPoint(0,0){O};
\tkzDrawSegments[->]([shift={(2,1)}]O,[shift={(-2,1)}]A);
\tkzLabelSegment[sloped]([shift={(2,1)}]O,[shift={(-2,1)}]A){$V^T$};
\tkzDrawSegments[->]([shift={(2,1)}]A,[shift={(-2,1)}]B);
\tkzLabelSegment[sloped]([shift={(2,1)}]A,[shift={(-2,1)}]B){$\Sigma$};
\tkzDrawSegments[->]([shift={(2,1)}]B,[shift={(-2,1)}]C);
\tkzLabelSegment[sloped]([shift={(2,1)}]B,[shift={(-2,1)}]C){$U$};
\draw[->] ([shift={(1,-1)}]O) to [out=-30,in=-150] ([shift={(-1,-1)}]C);
\node at (7.5,-2.5) {$A$};
% Create the labels:
\end{tikzpicture}

  }
\end{figure}

%%% Local Variables:
%%% mode: latex
%%% TeX-master: "main"
%%% End:
