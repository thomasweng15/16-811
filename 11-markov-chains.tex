\chapter{Markov Chains}
Let's first recall the definition of independent trials.
\begin{definition}[Independent trials]
A set of possible outcomes $E_1,E_2,...$ is given. With each outcome $E_k$ there is associated a probability $p_k$. The probability of a sampled sequence is defined as
$$ P\{(E_{j_0},E_{j_1},...,E_{j_k})\} = p_{j_0} p_{j_1} \hdots p_{j_k}$$
\end{definition}

\medskip
\noindent In the theory of Markov Chains the outcome of any trial depends on the outcome of the directly preceding trial only. 

\medskip
\noindent \textbf{Conditional probability}
\begin{itemize}
    \item $p_{jk}$: given that $E_j$ has occurred at some trial the probability of $E_k$ at the next trial.
    \item $a_k$ : probability of $E_k$ at the initial trial 
\end{itemize}
For instance here are the prob. of some sample sequences.
\begin{align*}
    &P\{(E_j,E_k)\} = a_j p_{jk}\\
    &P\{(E_j,E_k,E_r) \} = a_j p_{jk} p_{kr}\\
    \textrm{(and generally) }& P\{(E_{j_0},E_{j_1},...,E_{j_n})\} = a_{j_0} p_{j_0 j_1} p_{j_1 j_2} \hdots p_{j_{n-1} j_n}
\end{align*}

\section{Random Walk}
A random walk is the process by which randomly-moving objects wander away from where they started.
The simplest random walk to understand is a 1-dimensional random walk. Suppose that 
Given a set of events: 
$$ \{\dots -3, -2, -1, 0, 1, 2, 3, \dots\}$$
we have $P_{j,k} = 0$ if $ | j - k | > 1$. For a symmetric random walk we might accordingly have: $ P_{i, j} = 0$\\
$ P_{j, k} = \frac{1}{2}$ if $|j - k| = 1$
Graphically we would draw arrows with labelled probabilities. It is often a good way to think of a random walk as a markov chain as follows: 

\akash{Small figure here}

\begin{definition}
A sequence of trials with possible outcomes $E_1, E_2, \dots$ is called a Markov chain if the probabilities of sample sequeces are defined by: 
$$ P(E_{j_0}, E_{j_1}, \dots , E_{j_n}) = a_{j_0}p_{j_0, j_1} \dots p_{j_{n-1} j_n}$$ in terms of a probability distribution $\{a_k\}$ for $E_k$ at the initial (or zero-th) trial and fixed conditional probabilities $p_{j, k}$ of $E_k$ given that $E_j$ has occured in the preceding trial.
\end{definition}
where: $E_k$ = states of the system, $a_k$ = the probabilities of E_k at the initial trial $P_{j, k}$ the probabilities of the transition from $E_j$ to $E_k$

Furthermore, the matrix (finite or infinite) of transition probabilites: 
\begin{align*}
    \mathbf{P} = \begin{bmatrix} 
                    P_{11} & P_{12} & P_{13} & \dots \\
                    P_{21} & P_{22} & P_{23} & \dots \\
                    P_{31} & P_{32} & P_{33} & \dots \\
                    \vdots & \vdots & \vdots & \ddots \\
                \end{bmatrix}
\end{align*}
is a square matrix with non-negative elements and unit row sums. Such a matrix is called a \underline{Stochastic Matrix}.

Any stochastic matrix 