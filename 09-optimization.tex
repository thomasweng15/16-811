\chapter{Optimization}

Worked on by Alex.

This is another vast and deep subject, and we will only take a brief peak.
We have of course already seen some forms of optimization, in the form of error minimization.
No doubt you are familiar as well with the optimization of linear systems.
In this set of notes we will focus on optimizing arbitrary non-linear functions.
And in the next section of notes we will consider optimization of functionals, that is, functions of functions.

\section{Applications (Let's focus on robotics-related optimization)}

\begin{itemize}
  \item Minimize energy/power consumption.
  \item Minimize torque, null space forces.
  \item Path planning via potential fields.
  \item Computation of stable resting configurations.
  \item Maximize sensing information. (e.g. where to probe next?)
  \item Optimize execution time (Actually, this leads to Markov Decision Theory, Dynamic Programming, and Calculus of Variations. We will discuss C of V in detail later.)
\end{itemize}

We will now become increasingly more analytical and less numerical.
You should also put on your geometric thinking caps, since much of the analysis is best understood in terms of geometry.

The basic problem: Given a function of $n$ variables
\begin{align}
  f: \mathbb{R}^n \rightarrow \mathbb{R}
\end{align}
we would like to find its minimum (or maximum), if it exists.
In many cases it is difficult to find a global minimum, so we will often settle for local minima.

Recall how this process works in one dimension ($n = 1$).

We have a function $f: \mathbb{R} \rightarrow \mathbb{R}$. First we might compute the critical set of $f$,
\begin{align}
  C_f = \{ x \mid f'(x) = 0 \}
\end{align}
By examining this set we can determine those $x$ that are global minima.

Notice that computation of $C_f$ seems to entail finding the zeros of a function, namely the zeros of $f$.
In other words, we have reduced the optimization problem to the root-finding problem.
So, I guess we are done.
Well, not quite.
In higher dimensions, it is often easier to find (local) minima than one would think, based on our experience in finding simultaneous zeros in higher dimensions.
Intuitively, this is because $f'$ is not an arbitrary function, but rather a derivative.

We will thus have a lot more to say about optimization in higher dimensions than we did about root-finding.

\section{Golden Section Search (analogy to bisection)}

\underline{Basic idea}: Want to bracket a minimum, then shrink the bracket.

The problem is that we don't know the value of the function at the minimum, so we can't know whether we have bracketed a minimum.
Fortunately, if we are willing to settle for a local minimum, then we really just want to bracket a zero of $f'$.
That we can do, by a numerical approximation to the derivative of $f$.
The inner part of the loop works as follows.

We have three points $a$, $b$, $c$ such that $a < b < c$ and
\begin{align}
  f(b) < \min \{ f(a),\,\,f(c)\}
\end{align}
E.g.
\alex{Example figure here}

Now choose a point $x$, say halfway between $a$ and $b$.

If $f(x) > f(b)$, then the new bracketing triple becomes $[x, b, c]$.

If $f(x) < f(b)$, then the new bracketing triple becomes $[a, x, b]$.

See Numerical Recipes in C (NRiC) pg. 294 for wise words regarding limitations on how small you can make the bracketing interval.
And see the rest of NRiC for other methods, e.g. Parabolic Interpolation.

\subsection{Now, let's turn to the $n$-dimensional case}

First, recall the following conditions for a relative (local) minimum.

\section{Global Minima}

\section{Steepest Descent}

\section{Conjugate Gradient Method}

\section{Newton's Method}

\section{Constrained Optimization}

\section{Omitted Topics}
