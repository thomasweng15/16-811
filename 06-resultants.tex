\chapter{Resultants and Elimination Theory}


Worked on by Anirudh.

One technique with several variations.

\section{Isolating simultaneous zeros}
\anirudh{Figure here}

$p(x, y)$ and $q(x, y)$ are bivariate polynomials in $x$ and $y$. We
seek their simultaneous zeros. The method of resultants produces a
single univariate polynomial $R(x)$ such that
\begin{align*}
  R(x) = 0
\end{align*}
if and only if $p(x, y) = 0$ and $q(x, y) = 0$ for some $y$. So, find
the roots of $R(x)$. For each of these, solve a polynomial in $y$ to
get the simultaneous roots of $p$ and $q$.

\textbf{Application:} One can use this technique to split a robot's
configuration space into critical sections within which the topology
does not change -- Algebraic Cylindrical Decomposition.

\section{Singular simultaneous zeros}

The method tells us when an overconstrained algebraic system has a
simultaneous zero. E.g. $p(x) = 0$ and $q(x) = 0$ (2 equations, 1
unknown).

\anirudh{Figure here}

This is parameter elimination, i.e. the parameter $x$ is
``removed''. Really, what we get is a resultant $R(\coeff
\pandq)$ that tells us whether or not $p$ and $q$ have a
simultaneous root. If $R(\coeff\pandq)= 0$ then yes,
otherwise no. If the coefficients are known numbers, then we just plug
in and test. If the coefficients are symbols, then the equation
$R(\coeff\pandq) = 0$ provides constraints on the coefficients that
tell us the conditions under which a simultaneous zero exists

\section{Implicitizing parametric equations}

\anirudh{Figure here}

Suppose we have a parameterized curve in $2$D
\begin{align*}
  \alpha(t) = (p(t), q(t))
\end{align*}
with $p$ and $q$ polynomials in $t$. We might want an implicit
equation $F(x, y) = 0$ for the same curve, with $F$ a polynomial in
$x$ and $y$.

Elimination theory will give this to us, basically by constructing the
system of equations
\begin{align*}
  p(t) - x &= 0 \\
  q(t) - y &= 0
\end{align*}
If we think of this as $2$ equations in $1$ unknown $t$, then we are
back in the ``singular simultaneous zeros'' scenario. In other words,
we have two polynomial equations in $t$. We think of $x$ and $y$ as
part of the coefficient set of these equations. The result is a
polynomial equation $F(x, y) = 0$ that provides constraints on $x$ and
$y$ for a simultaneous zero in $t$. In other words, it gives us  our
desired implicit equation.

\section{Brief intro to resultants}

\section{Sylvester's Method}


%%% Local Variables:
%%% mode: latex
%%% TeX-master: "main"
%%% End:
