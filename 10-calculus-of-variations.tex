\chapter{Calculus of Variations}
Worked on by Ratnesh

The calculus of variations arose in tight complicity with the development of mechanics. 
Much of the axiomatic grounding of physics in general and mechanics in particular consists of variational principles. 
Indeed, the rudiments of quantum theory may be derived from Newton's laws and variational principles. 
We will in these notes trace some of the history of the calculus of variations. 
One caution: while much of the treatment will seem mathematical, it is in fact very handwavy. 
A true and correct treatment would require us to pull out such differential geometry tools as covariant derivatives, affine connections, and n-forms. 

Applications:
\begin{itemize}
    \item Path optimizational
       \begin{itemize}
            \item Optimal control (minimum cost trajectories)
        \end{itemize}
    \item Engineering
        \begin{itemize}
            \item Vibrating membranes
            \item Theory of elasticity
            \item Electrostatistics
        \end{itemize}
    \item Machine Vision
        \begin{itemize}
            \item Surface reconstruction
            \item Image flow (Motion and Structure from Optical Flow)
            \item Edge detection
        \end{itemize}    
\end{itemize}

\section{Introduction}
We have seen the basic principle ``To minimize $P$ is to solve $P' = 0$"

So far, we have only looked at finite-dimensional problems, that is, minimization of some function $f: \mathbb{R}^n \mapsto \mathbb{R}$. 
In such a problem, we seek the value of $n$ numbers that minimize $f$. 

What about infinite-dimensional problems, that is, problems in which $P$ depends on an infinity of numbers?

In particular, what about functionals (functions of functions)?

\bigbreak

\textbf{Example}:\\
Suppose we connect two points in the plane, $(x_0, y_0)$ and $(x_1, y_1)$ by a rectifiable curve of the form $y = y(x)$. 



\begin{figure}
    \centering
    \includegraphics[width=\textwidth ]{figures/placeholder.png}
    \caption{bla.}
    \label{fig:}
\end{figure}
