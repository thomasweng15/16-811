\chapter{Differential Geometry: Frenet Frames and Surface Curvature}

\begin{remark}
  Material largely adopted from ``Elementary Differential Geometry''
  by B. O'Neill
\end{remark}

Differential geometry studies the motions possible in a space. Some
key concepts:

\begin{definition}[Tangent Vector $v_p$]
  A vector anchored at a particular
point $p$. Set of all possible $v_p$ for a given $p$ is called the tangent
space $T_p$ at $p$.
\end{definition}

\anirudh{Small figure here}

\begin{definition}[Tangent Bundle]
  A space along with all its tangent vectors.
\end{definition}


\begin{example}
  If $\Real^n$ is the underlying space then we have another $\Real^n$
  at each point $p \in \Real^n$ consisting of all the tangent vectors
  anchored at $p$. So we get $\Real^n \times \Real^n$ over all, just
  like a state space. For $n = 2$:
  \anirudh{Example figure here}
\end{example}

\begin{example}
  The tangent bundle associated with a circle looks like
  $\Sphere^1 \times \Real^1$:
  \anirudh{Example figure here}
\end{example}

\textbf{Vector field} A function $M \rightarrow T(M)$ where $M$ is the
underlying space, often called a \textit{manifold} (e.g. $\Real^n$,
$\Sphere^n$, even matrix groups e.g. SO(n)), and $T(M)$ is the tangent
bundle. For a point $p$ it defines a mapping $p \rightarrow v_p \in
T_p$ often denoted by $V(p)$ or $V_p$ (We generally require $V$ to be
smooth, meaning as many derivatives as we need.)

One classic question is whether a manifold has a continuously varying
vector field that is never zero. 
