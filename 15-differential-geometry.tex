\chapter{Differential Geometry: Frenet Frames and Surface Curvature}

\begin{remark}
  Material largely adopted from ``Elementary Differential Geometry''
  by B. O'Neill
\end{remark}

Differential geometry studies the motions possible in a space. Some
key concepts:

\begin{definition}[Tangent Vector $v_p$]
  A vector anchored at a particular
point $p$. Set of all possible $v_p$ for a given $p$ is called the tangent
space $T_p$ at $p$.
\end{definition}

\anirudh{Small figure here}

\begin{definition}[Tangent Bundle]
  A space along with all its tangent vectors.
\end{definition}


\begin{example}
  If $\Real^n$ is the underlying space then we have another $\Real^n$
  at each point $p \in \Real^n$ consisting of all the tangent vectors
  anchored at $p$. So we get $\Real^n \times \Real^n$ over all, just
  like a state space. For $n = 2$:
  \anirudh{Example figure here}
\end{example}

\begin{example}
  The tangent bundle associated with a circle looks like
  $\Sphere^1 \times \Real^1$:
  \anirudh{Example figure here}
\end{example}

\begin{definition}[Vector Field]
A function $M \rightarrow T(M)$ where $M$ is the
underlying space, often called a \textit{manifold} (e.g. $\Real^n$,
$\Sphere^n$, even matrix groups e.g. SO(n)), and $T(M)$ is the tangent
bundle. For a point $p$ it defines a mapping $p \rightarrow v_p \in
T_p$ often denoted by $V(p)$ or $V_p$ (We generally require $V$ to be
smooth, meaning as many derivatives as we need.)  
\end{definition}

One classic question is whether a manifold has a continuously varying
vector field that is never zero.

We just drew one for $M = \Sphere^1$. It is impossible for $M =
\Sphere^2$ (of course there is a 2D tangent space $T_p$ for each $p
\in \Sphere^2$, but we can't find a function $V: \Sphere^2 \rightarrow
T(\Sphere^2)$ that is nonvanishing and continuous. Proof is beyond
these lectures; it entails studying antipoal maps and fixed point
theorems.)

When we have one manifold embedded in another we can ask whether it is
possible to find smooth \underline{unit normal vector fields}
(assuming we have defined an inner product). For example, for the
circle in the plane it is possible for orientable submanifolds (this
leads to ideas like the Gauss map and Gaussian curvature.)
\anirudh{Example figure here}

\section{Geometry of Curves in $\Real^3$}
\label{sec:geom-curv-real3}

We will consider parameterized curves and we will assume that they are
sufficiently smooth to give us as many derivatives as we need
(e.g. $\continuous^3$)

\begin{definition}[Curve]
  A curve is a smooth function $\alpha: I \rightarrow \Real^3$, with $I$
some interval in $\Real^1$. We often write $\alpha(t) = (\alpha_1(t),
\alpha_2(t), \alpha_3(t))$.
\end{definition}

\begin{definition}[Velocity Vector]
  The velocity vector of $\alpha$ at time $t$ is the tangent vector of
  $\Real^3$ given by $\alpha'(t) = (\alpha_1'(t), \alpha_2'(t),
  \alpha_3'(t))$. 
\end{definition}

\begin{definition}[Speed]
  The speed of $\alpha$ at time $t$ is $v(t) = \|\alpha'(t)\|$
\end{definition}

\begin{definition}[Arclength traversed]
  The arclength traversed between time $t_0$ and time $t_1$ is
  \begin{equation*}
    \int_{t_0}^{t_1} v(t) dt
  \end{equation*}
\end{definition}

\begin{theorem}
  Suppose $\alpha: [a, b] \rightarrow \Real^3$ is a curve for which
  $\alpha'(t)$ is not ever $0$ on $[a, b]$. Then one can
  reparameterize $\alpha(t)$ as $\beta(s)$ with $s$ measuring
  arclength. Note that $\beta$ gives a unit-speed parameterization of
  the curve. (So $\beta(s) = \alpha(t(s))$ where $t(s)$ is the time at
  which the curve $\alpha$ would have reached traversed arclength $s$)
\end{theorem}
\begin{proof}
  Define $s(t) = \int_{a}^t \|\alpha'(u)\| du$, for $t \in [a,
  b]$. Then $s'(t) = \|\alpha'(t)\| > 0$. So $s(t)$ is strictly
  monotone, meaning the inverse $t(s)$ exists (not always easy to
  calculate, of course.)

Let $\beta(s) = \alpha(t(s))$, $s \in [0, s(b)]$

Note that
\begin{align*}
  \beta'(s) &= \frac{d}{ds} \alpha(t(s)) \\
&= \alpha'(t(s)) \frac{dt}{ds}(s)
\end{align*}
So $\|\beta'(s)\| = \|\alpha'(t(s))\| \frac{dt}{ds}(s)$ since
$\frac{ds}{dt} > 0$, $\frac{dt}{ds} > 0$.
Thus, $\|\beta'(s)\| =
\frac{ds}{dt}(t(s)) \frac{dt}{ds}(s) = 1$.

\end{proof}

\begin{example}[A helix in $\Real^3$]
  $\alpha(t) = (r\cos t, r\sin t, qt)$, $r > 0$, $q \neq 0$ where
  $(r\cos t, r\sin t)$ is the circular part, $qt$ is the rise/fall,
  and $t \in [0, \infty]$ for simplicity (but could use any interval
  of course, including $[-\infty, \infty]$ but the might also want
  arclength to go from $-\infty$ to $\infty$; just break into two
  parts joined at $t = 0$/$s = 0$)
  \anirudh{helix figure here}
  \begin{align*}
    \alpha'(t) &= (r\sin t, r\cos t, q) \\
    v(t) &= \sqrt{r^2 + q^2} = c \\
    s(t) &= \int_0^t c du = ct
  \end{align*}
  where $c$ is the constant speed. Thus $t(s) = \frac{s}{c}$. So we
  can reparameterize as
  \begin{align*}
    \beta(s) = \alpha(\frac{s}{c}) = (r\cos \frac{s}{c}, r\sin
    \frac{s}{c}, \frac{qs}{c})
  \end{align*}
  Now we have a unit speed curve giving the same shape (Not usually so
  easy to reparameterize this way.)
\end{example}

\underline{\textbf{Observe:}} Suppose we have a curve $\alpha: I
\rightarrow \Real^3$ and a smooth in $t$ function that assigns to each
point $\alpha(t)$ a vector $V(t)$ of $\Real^3$ ($V(t)$ need not be
tangent to $\alpha(t)$, merely a tangent vector of $\Real^3$.)
Differentiating $V(t)$ allows us to obtain information about
$\alpha$.

Let's first look at a unit-speed curve $\beta: I \rightarrow \Real^3$
(so $\beta(s)$ is a parameterization in terms of arclength $s$.)
Define the following three vector fields on $\beta$: $T = \beta'$
called the \textit{unit tangent vector field} of $\beta$, $N =
\frac{T'}{\|T'\|}$ called the \textit{principal normal vector
  field} of $\beta$, and $B = T \times N$ called the
\textit{binomial vector field} of $\beta$.
The quantity $\kappa(s) = \|T'(s)\|$ also has a name,
\textit{curvature function} of $\beta$.


\underline{\textbf{Note:}} Since $\beta$ is unit-speed, $T$ is a unit
vector. It could be that $T' = 0$, in which case $N * B$ are not
well-defined. This occurs for instance when $\beta$ is a straight
line, or when it is instantaneously linear. So, let's assume $\kappa >
0$ over the entire curve segment $I$ that we are considering.

%%% Local Variables:
%%% mode: latex
%%% TeX-master: "main"
%%% End:
